\documentclass[12pt]{article}
%\usepackage{times}
\usepackage{cite}
%this is a comment
\title{Data Visualization for Civic Engagement in Corvallis City}
\author{Khuong Luu (luukh) and Aidan Grimshaw (grimshaa)}




\begin{document}
\maketitle
\tableofcontents


\section{Why do we need a (better) visualization for civic data?}

\subsection{The problem of low civic engagement and government transperancy}

Although voter turnout rates are relatively high for presidential races (50-60\%), there is relatively little turnout in local elections [1]. An average of 15\% of Americans turnout for these elections. Some cities, like Las Vegas, Fort Worth, and Dallas have turnout rates below 10\%, at 9.4\%, 6.5\%, and 6.1\% respectively [2].

Low voter turnout is a symptom of a larger problem. Eligible voters do not currently have access to the information they need to find issues that they can engage with. In a 2015 LA Times poll,  39\% of Los Angeles residents said that they didn’t get involved in the community because they were not sure what they could do to help [5].

The root cause of this issue is a lack of accessibility of government data.

As data science and data focused investigation has matured, there are many news organizations like fivethirtyeight and vox that have started to use data to tell stories and lend new insights to their readers.

This contrasts sharply with the way government data is currently displayed. Most government information available online, but is in a format inaccessible to the casual voter. FEMA for example, has all of their flood insurance map data available online, but the data comes with a 57 page tutorial required to understand it [4]. Many smaller local governments like Corvallis don’t have government data posted on their websites.

Government data is about accountability. By making information being inaccessible to the average voter, local governments make their elected officials less accountable both directly and indirectly. Elected officials are less accountable directly by the lack of objective statistical measures of their performance, and indirectly, through lack of voter turnout. This is a problem when we want the government to represent the people.

\subsection{Current existing solution is doing it poorly for the common people}

Upon our research, we found that there are some existing solutons available such as city-data.com, corvallisoregon.gov, data.orcities.org. However, these solutions is doing it poorly. city-data.com design is not user friendly, slow loading, and not useful visualization; corvallisoregon.gov only provide plain data in form of long, hard to comprehend text file, written in government language, data.orcities.org only provide API and so not accessible to the common people. Imagine I middle age person with minium technology experience want to find out more about how our goverment, neiborhood, city is doing, how could he/she nagivigate through all of these mess?


\section{Our solution}

Our website accquires data from many available sources (and also maybe more data from the city if needed, with the help of the school), cleaning them, and visualize them in a interactive way and easy to comprehend with user experience in mind. Through that, we inform people in Corvallis (or people who are interested) a clear civic information that can make them make better voting decision, more aware of the problems and challenge of the city, and the neiborhood they are living in.In the other hand, the city government may also be able to make better governing decision.

The key differences between our website and some existing websites such as city-data.com and opendatanetwork.com are that: First, we are more focus on Corvallis city and have connection with the city. Second, we're better in city-data.com in that we visualize data more interactively, more attractive. Third, our website make data more accessible compared to goverment website like corvallisoregon.gov which only provide some obscure pdf files. Fourth, existing solutions only show data of a few recent years, we aims to make more data available. 

What make our application/website uniquely effective and useful are that our visualization is interactive and is designed with the user experience in mind, our application will run faster, have longer data range, and we might be able to invite domain experts in the topic we visualize to give inbiased insights. These features make civic engagement more accessible 

\section{Development Roadmap}

One of the limiations of our application is the time to complete the project. We have only 10 weeks and not all of us are familiar with the technologies we're going to use to build this application. Thus, we have to propose small, start small and build the solution incrementally, and try our best. Alongside, we can help each other in technical difficulities as a group.

The single most serious challenge I see in developing the product on schedule is the time we need to accquire data from the government if the data we need is not available or we alone can not get permission to accquire the data but also need to help of OSU in the scope of student project as we are not a credible entity to "talk" to the government of Corvallis. When crafting this ideas, the authors are certainly not totally 100\% sure about if we can get enough data we need but we are certain we can eventually get it or adjust our goals a little bit. In all, the problem is how long it would take to get enough data, and that duration could be a block to our project planning and development. Our solution is to start looking and asking for data from the city as soon as our project is approved so that we will hopefull have enough time. While waiting to get the real data, we will build our application with sample data first so we don't have to wait until the real data arrives. In this way, we can minize the risk of project failure. Furthermore, we also event have a backup plan. If in the worst case we can't get enough data we need from the city, we will change our subject to Oregon State University. This subject is much easier and faster to accquire data.

\section{References}


[1]"In the U.S., Almost No One Votes in Local Elections", CityLab, 2018. [Online]. Available: https://www.citylab.com/equity/2016/11/in-the-us-almost-no-one-votes-in-local-elections/505766/. [Accessed: 16- Jan- 2018].

[2]"Who Votes For Mayor?", Whovotesformayor.org, 2018. [Online]. Available: http://www.whovotesformayor.org/. [Accessed: 16- Jan- 2018].


[3]N. Lelyveld, "Poll finds L.A. County residents want to be more involved in their community but don't have time", latimes.com, 2015. [Online]. Available: http://www.latimes.com/politics/la-pol-ca-vision-poll-la-county-story.html. [Accessed: 16- Jan- 2018].

[4]"How To Read a Flood Insurance Rate Map Tutorial | FEMA.gov", Fema.gov, 2018. [Online]. Available: https://www.fema.gov/media-library/assets/documents/7984. [Accessed: 16- Jan- 2018].


\end{document}
